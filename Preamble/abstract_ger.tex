\addchap*{Kurzzusammenfassung}
\label{kurzzusammenfassung}
{\LARGE Verifikation von Visualisierungen von komplexen Avionik Modellen mit Computer Vision}

Mit der steigenden Komplexit{\"a}t von Modellen und Applikationen wird die Nutzung von Domain Specific Modeling (DSM) in der Luftfahrtindustrie immer wichtiger. Es erm{\"o}glicht Ingenieuren, effizienter und sicherer zu arbeiten und reduziert duch autmatische Code Generation die Fehleranf{\"a}lligkeit der fertigen Programme. Bei sicherheitskritischen Anwendungen jedoch ist DSM durch die n{\"o}tige Verifikation der Modell-Visualisierungen mit signifikantem Mehraufwand verbunden.\\
Diese Arbeit baut auf die Ergebnisse von \cite{og_paper} auf, um die Zuverl{\"a}ssigkeit der automatischen Verifikation von Block-Diagramm-Visualisierungen im DSM zu verbessern. In \cite{og_paper} werden Methoden der Computer-Vision aus der Python-Bibliothek \textit{OpenCV} genutzt, um Block-Diagramm-Modelle automatisch zu erkennen, mit den urspr{\"u}nglichen Modellen zu vergleichen und auftretende Abweichungen innerhalb des grafischen browser-basierten Modelleditors \textit{eXtensible Graphical EMOF Editor} oder \textit{XGEE} f{\"u}r den Benutzer zu visualisieren. Die urspr{\"u}ngliche Implementierung der Blockdiagramm-Erkennung war jedoch begrenzt in ihrer F{\"a}higkeit, komplexe und vielf{\"a}ltige Diagrammtypen innerhalb von XGEE korrekt zu verarbeiten.

Diese Arbeit erweitert die bestehende Implementierung der Blockdiagramm-Erkennung, um die Erkennung von komplexen Diagrammen dreier Typen innerhalb von XGEE zu erm{\"o}glichen. Die neue Implementierung verwendet eine Kombination von Methoden aus der Computer-Vision, um Kreuzende oder teilweise verdeckte Verbindungslinien in verschiedenen Ausrichtungen, diverse Vertices in unterschiedlichen Gr{\"o}{\ss}en und Anordnungen sowie Textbl{\"o}cke in unterschiedlichen Orientierungen zu erkennen.\\
Diese Verbesserungen demonstrieren das Potenzial von Computer-Vision-Methoden, die Verifikation von DSM-Modellen zu automatisieren und die Sicherheit in sicherheitskritischen Anwendungen der Luftfahrt zu erh{\"o}hen.